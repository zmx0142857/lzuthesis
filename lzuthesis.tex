\documentclass{lzuthesis}

% 中文标题
\zhtitle{兰州大学本科生毕业论文模板}
% 英文标题
\entitle{Lanzhou University Thesis Template for Bachelor Degree}
% 支持断行的中文标题 (用于封面)
%\def\zhtitleMultirow{兰州大学本科生毕业论文模板\newline 兰州大学本科生毕业论文模板}
% 支持断行的英文标题 (用于封面)
\def\entitleMultirow{A Simple \LaTeX\ Template of Lanzhou University Thesis \newline for Bachelor Degree}
% 短标题 (用于页眉)
\shorttitle{兰大本科毕设模板}
% 作者中文名
\zhauthor{卓铭鑫、黄洋锐}
% 作者英文名
\enauthor{Zhuo Mingxin, Huang Yangrui}
% 单位中文名
\zhinstitute{兰州大学数学与统计学院}
% 单位英文名
\eninstitute{School of Mathematics and Statistics, Lanzhou University}
% 导师
\teacher{指导老师}
% 学院
\college{数学与统计学院}
% 专业
\major{数理基础科学}
% 年级
\grade{2016}

\begin{document}

% 封面
\makecover

% 诚信责任书
\makeliability

% 中文摘要
\begin{cnabstract}
	本文是根据兰州大学教务处《兰州大学本科毕业论文(设计)写作规范》(2018)
	(\url{http://jwc.lzu.edu.cn/upload/news/N20180928090802.doc},
	下称《手册》) 要求所编制的 \LaTeX\ 模板. 具体使用中请根据实际情况更改.
\end{cnabstract}
\cnkeywords{毕业论文; \LaTeX{}; 模板}

% 英文摘要
\begin{enabstract}
	This article is a \LaTeX\ template based on this file from the Academic Affairs Office of LZU: \url{http://jwc.lzu.edu.cn/upload/news/N20180928090802.doc}. You may modify this template for your own use.
\end{enabstract}
\enkeywords{thesis; \LaTeX{}; template}

% 目录
\maketoc

% 引言
%\section{引言}
%介绍研究背景.

% 正文
\section{使用方法}

\subsection{本模板包含的文件}

\begin{itemize}
	\item 帮助文件 \texttt{lzuthesis-original.pdf}: 你正在看的文件;
	\item 模板文件 \texttt{lzuthesis.cls}: 定义了本文所用的模板.
	封面页, 诚信责任书, 致谢和评语等部分均已在该文件中按《手册》要求实现;
	\item 论文源文件 \texttt{lzuthesis.tex}: 用户只需按提示修改它即可;
	\item 文件夹 \texttt{imgs/}: 包含兰州大学校徽和校名的矢量图形,
		用户可以把自己的图片放在这里.
\end{itemize}
\texttt{lzuthesis.cls} 和 \texttt{lzuthesis.tex}
中的注释应该能帮助用户进行简单修改.

\subsection{编译}

推荐使用最新的 texlive 发行版和 texstudio 编辑器.
(texlive 是开源项目, 可通过清华镜像下载; texstudio 为免费软件).
确保你使用的编译器是 xelatex, 而不是 pdflatex. 如果用命令行编译,
应当编译两次以形成正确的目录和引用:

\begin{verbatim}
$ xelatex lzuthesis.tex
$ xelatex lzuthesis.tex
\end{verbatim}

\subsection{字号字距}

全文字号字距按照《手册》实现, 如有需要, 可用 \verb|\zihao{n}| 调整字号. 其中 n = 0 到 8 分别对应初号到八号, n = -0 到 -6 分别对应小初到小六. \verb|\ziju{n}| 用于调整字距, 两字之间相距 n 个字的空白.

\subsection{宏包}

本模板包含的宏包如下, 在编译过程中提示缺少宏包请自行下载:

\begin{verbatim}
amsmath     enumitem    hyperref    titlesec
amssymb     fancyhdr    listings    tocloft
appendix    float       mathrsfs    xcolor
array       geometry    subcaption
ctex        graphicx    tabularx
\end{verbatim}

\subsection{发布}

由于作者能力所限, 模板中可能仍然存在着未发现的缺陷, 如有更改请加在
\texttt{lzuthesis.cls} 的[修改日志]区域并重新压缩分享. 

\section{示例}

\subsection{定理与证明}

本文提供了常见的定理与证明环境:

\begin{verbatim}
definition  定义    axiom       公理    lemma       引理
theorem     定理    proposition 命题    corollary   推论
property    性质    example     例      remark      注
proof       证明    solution    解
\end{verbatim}

\begin{theorem}[\textbf{Banach 不动点定理--压缩映像原理}]
	\label{thm1}
	设 $(\mathscr{X}, \rho)$ 是一个完备的距离空间, $T$ 是 $(\mathscr{X},
	\rho)$ 到其自身的一个压缩映射, 则 $T$ 在 $\mathscr{X}$
	上存在唯一的不动点.
\end{theorem}

\begin{proof}
	略, 详见泛函分析课本. 
\end{proof}

\subsection{图片}

\begin{figure}[!h]
	\centering
	\includegraphics[width=0.2\textwidth]{lzu.pdf}
	\caption{单张图片}
	\label{p1}
\end{figure}

\begin{figure}[!h]
	\centering
	\begin{minipage}[t]{0.2\linewidth}
		\centering
		\includegraphics[width=\linewidth]{lzu-blue.pdf}
		\caption{左主图}
	\end{minipage}
	\quad
	\begin{minipage}[t]{0.2\linewidth} 
		\centering 
		\includegraphics[width=\linewidth]{lzu.pdf} 
		\caption{右主图} 
	\end{minipage}
\end{figure}

\begin{figure}[!h]
	\centering
	\begin{subfigure}[b]{0.2\linewidth}
		\includegraphics[width=\linewidth]{lzu-blue.pdf}
		\caption{左子图}
	\end{subfigure}
	\quad
	\begin{subfigure}[b]{0.2\linewidth}
		\includegraphics[width=\linewidth]{lzu.pdf}
		\caption{右子图}
	\end{subfigure}
	\caption{\texttt{subfigure} 环境需要 \texttt{subcaption} 包}
\end{figure}

\subsection{表格}

\begin{table}[!h]
	\caption{三线表}
	\centering
	\begin{tabular}{ccc}
		\hline
		a &b &c \\
		\hline
		1 &2 &3 \\
		1 &2 &3 \\
		\hline
	\end{tabular}
\end{table}

\begin{table}[!h]
	\caption{定宽表可以指定表格宽度, 超出部分自动折行. \texttt{tabularx} 环境需要 \texttt{tabularx} 包}
	\label{定宽表}
	\centering
	\begin{tabularx}{0.6\linewidth}{c|X}
		\hline
		1 & balabala  balabala  balabala  balabala  balabala  balabala  balabala  balabala  balabala \\
		\hline
		2 & balabala  balabala  balabala  balabala  balabala  balabala  balabala  balabala  balabala \\
		\hline
	\end{tabularx}
\end{table}

\subsection{交叉引用}

引用正文图片\ref{p1}; 引用附录图片\ref{pp1};
引用表\ref{定宽表};
引用公式\ref{eq1};
引用定理\ref{thm1};
引用文献\upcite{bib1};
使用脚注\footnote{脚注内容}.

\section{推荐}

本模板提供给有一定 \LaTeX\ 基础的同学, 如果需要 \LaTeX\ 入门, 可以参考
\url{https://github.com/zmx0142857/man/} 下的 \verb|tex/| 目录.
附 \LaTeX\ 或 \TeX\ 的参考网站:
\begin{enumerate}
	\item \url{http://www.ctan.org}, \TeX\ 资源集散网站, 可从该网站下载最新宏包.
	\item \url{http://tex.stackexchange.com}, \LaTeX\ 和 \TeX\ 的国际交流社区.
\end{enumerate}

% 结论
%\section{结论}
%主要的研究结果.

% 参考文献
\begin{bibliography}
	\bibitem{bib1} 作者. 文章题目 [J].  期刊名, 出版年份,卷号(期数):
		起止页码.
	\bibitem{bib2} 作者. 书名 [M]. 版次.
		出版地: 出版单位,出版年份: 起止页码.
\end{bibliography}

% 附录
\begin{appendix}
	
\section{附录标题}

\begin{figure}[!h]
	\centering
	\includegraphics[width=0.3\textwidth]{lzu.pdf}
	\caption{附录的图片}
	\label{pp1}
\end{figure}

一维坐标系上的热传导方程
\begin{equation}
	\label{eq1}
	\frac{\partial u(x,t)}{\partial t}
	= a\frac{\partial ^{2}u(x,t)}{\partial x^{2}}
\end{equation}

\begin{lstlisting}[language=python, caption={helloworld.py}]
print("Hello World")
\end{lstlisting}
	
\end{appendix}

% 致谢
\begin{acknowledgement}
	感谢为我们的模板提供建议和问题反馈的邓儒嘉、伍文超、黄建川等同学.
	感谢 ctex 宏包的编写人员. 感谢 D.~E.~Knuth, 他创造的 \TeX\
	语言既是艺术也是折磨.
\end{acknowledgement}

% 评语
\makereviews

\end{document}
